\documentclass{aamas2012}

\usepackage{cancel}

\pdfpagewidth=8.5truein \pdfpageheight=11truein

\newtheorem{theorem}{Theorem}
\newtheorem{definition}{Definition}
\newtheorem{corollary}{Corollary}

\begin{document}

\title{Optimal Broadcast Auctions with Costly Actions}

\numberofauthors{2}

\author{ \alignauthor XXX \alignauthor YYY }


\maketitle

\section{Introduction}

\section{Cost Model and Settings}

In this section, we define our cost model and explain why they make sense. Our
cost model is inspired from online second-hand item transactions, including
examples like eBay, craigslist and universities' mailing-lists.


\begin{definition}\label{def:model}

In our settings, there's only one seller selling one item to $n$ i.i.d.
(indenpendent and identically distributed) bidders whose valuation distribution
(CDF) is continuous over $[0, 1]$ (i.e. no mass point).  The seller can
broadcast a message to all bidders with a cost $b$ (the broadcast cost). A
bidder may remain silent after that broadcast with no cost.  If not, it costs
$c_0$ if the bidder's bidding(reply) doesn't imply positive valuation.
Otherwise, if the bidding implies a positive valuation, a greater bidding cost
$c_1 > c_0$ is charged.  Such bidding cost $c$($c_0$ or $c_1$) main contain two
parts, $\beta_1, \beta_2$ where $\beta_1+\beta_2 = c$.  The first part
$\beta_1$ is charged to the seller while the second is charged to the
corresponding bidder. The bidder's reply should be deterministic with respect
to the seller's broadcast (for simplicity, we only consider pure strategy
equilibrium).

\end{definition}

(shall we insert a graph to show our cost model?)

Most settings in definition \ref{def:model} are quite standard except for the
cost and broadcast capability.

We introduce broadcast capability because it's shared by many auctions.  For
example, a Vickrey auction or a first-price auction with reserve price can be
described as a broadcast auction with only one broadcast: telling every bidder
the reserve price. The bisection auction [cite bisection auctions] is another
example which has many rounds of broadcasts. In each round , it will broadcast
a price and ask bidders to reply whether his valuation is beyond or below that
price.  In real world, sellers make such broadcasts via sending emails to a
bunch of receivers (typically a mailing-list), listing items on a platform such
as craigslist and eBay, or even making ads on Internet/TV. Such broadcast
activities costs either money (e.g. a list or ads fee) or time/effort (e.g.
writing and sending an email).

Note that in our model, we only give sellers broadcast capability so they
cannot find or communicate with each bidder one by one. The first reason to
have this constraint is there are too many potential bidders on the Internet
(our model focus on online item transactions) and it's hard to explicitly find
them one by one. On the other hand, in offline cases where the set of bidders
are small and explicit (e.g. the government want to sell a land to one of three
companies), it might be helpful to let the seller communicate with bidders one
by one [cite search mechanisms]. The second reason is that we want to focus on
mechanisms that avoids time consuming bargaining. Such feature is very
important as one of the most vital advantages of online transactions are their
convenience and the time consuming bargainning can ruin it.

We distinguish bidding costs $c, c_0$ because a trustworthy system need to
verify a bidding which implies positive valution. For example, the bidder may
have to input his credit card number and prepay an amount of money. Without
such verification for the bidder, a bidder may bid very high and refuse to pay
in the end. A verification for the seller might also be needed. For example, a
bidder may want to set up an appointment with the seller to check the item.
Setting up such an appointment might be costly because they need to dicuss time
and place via emails or phone calls.  Attending that appointment may also cost
travel fees and time. Therefore biddings that imply positive valuations cost
more. Biddings that won't imply positive valuation seem to be useless, but
sometimes they do exist and do have cost. For example, the no-bid in bisection
auctions (i.e. the reply that indicate someone's valuation is below a price)
cost $1$-bit for communication.

The cost of bidding may differ from other cases (e.g. the higher valuation a
bidding implies, the higher cost there might be). However we only distinguish
two cases above since we believe that the difference between other cases are
not so significant. For example, entering credit card number and prepaying
\$100 should cost almost the same effort as entering credit card number and
prepaying \$10.

Finally, we define optimal mechanisms to be the ones that maximze seller's
utility since when facing many different auction mechanisms, a rational seller
will choose the one that gives him maximum utility.

\begin{definition}

We say a mechanism is optimal if it gives the seller maximum utility (revenue)
which equals to all the value payed to this seller minus all the cost charged
to this seller. A class of mechanisms are optimal if they contain one optimal
mechanism.

\end{definition}

\section{Multi-round Vickrey Auctions}

This section describes a class of mechanisms which are based on current
emperical mechanisms widely used for online item transactions. We'll also
analyze their equilibria to prepare for optimality analysis in the next
section.  This will further give us insight why our cost model is reasonable.
Without the cost model, it makes no sense to use such complex mechanisms but in
reality people do use them or a simplified version of them.

A Multi-round Vickrey Auction (MVA) has multiple rounds of Vickrey
auctions with progressively decreasing reserve prices. This kind of auction
effectively occurs on eBay. The seller may set up a reserve price and let
buyers bid for this item. The proxy bidding functionality makes such an auction
equivalent to a Vickrey auction with a reserve price. If no buyers bid for a
given reserve price, the seller may lower the reserve price, which makes the
whole process equivalent to an MVA.

\begin{definition}(Multi-round Vickrey Auction, MVA)

In a Multi-round Vickrey Auction (MVA), there's a sequence of reserve prices
$r_k, r_{k-1}, \ldots, r_0$ where $r_k > r_{k-1}$. The seller creates a Vickrey
auction with a reserve price $r_{k-i}$ at time $i$ (or round $k-i$). In each
Vickrey auction, if only one buyer bids, he/she gets the item and pays reserve
price. Otherwise, the buyer with the highest bidding gets the item and pays the
second highest bidding.

\end{definition}

MVAs require Vickrey auctions (or equivalent English auctions) as basic steps.
In reality, however, such functionality won't always be provided by online
platforms such as craigslist. Thus a simplified version of MVA occur very often
in those platforms. People call it first-come first served which means for
every reserve price $r_i$, the first one who accept that price wins the item
and pays $r_i$ directly. This mechanism may loose revenue and social efficiency
as the person with lower valuation $p$ may get the item for $r_i$ while there's
someone else who is willing to pay a higher amount of $q$ where $r_i \leq p < q
< r_{i+1}$. We won't focus on this first-come first served mechanism because
it's harder to analyze analytically and it's inferior than MVAs in terms of
both sellers' utility and social welfare.

\subsection{Equilibrium}

It's obvious to see that whenever a bidder decides to bid, he/she must bid
truthfully. Thus the Bayesian Nash Equilibria (BNE) for MVAs can be described
as $k$ thresholds $a_k, a_{k-1}, \ldots, a_0$ where $a_i > a_{i-1}$. Whenever
a bidder's valuation for the item is greater than $a_i$, he/she is going to bid
in round $i$ whose reserve price is $r_i$.

For MVA, assume:
\begin{align*}
    P(x) &:= Pr(v_{1},v_{2},\ldots,v_{n-1}\leq x)\\
    p(x) &:= Pr(\max_{1 \leq j\leq n-1}v_{j}=x)=P'(x)
\end{align*}
the equations to determine $a_i$ are: 
\begin{align}\label{eq:MVA_eq}
    &r_0 = a_0 \mbox{ and }
    \forall i ~(1 \leq i \leq k),\nonumber\\
    &~~P(a_{i})(a_{i}-r_i) =
    \int_{a_{i-1}}^{a_{i}}(a_{i}-x)p(x)dx+P(a_{i-1})(a_{i}-r_{i-1})
\end{align}
Here $v_1, v_2, \ldots, v_{n-1}$ are valuations of the remaining $n-1$ bidders.
The following theorem describes the equilibrium of MVAs determined by equations above.

\begin{theorem}
In a MVA with reserve prices $r_k, r_{k-1}, \ldots, r_0$, there's a pure
strategy Bayesian Nash Equibibrium characterized by thresholds $a_k, a_{k-1},
\ldots, a_0$.  The bidder with valuation greater than $a_i$ (but not greater
than $a_{i+1}$) will bid in round $i$ with reserve price $r_i$. The relation
between $a_i$ and $r_i$ is:
\begin{align}\label{eq:MVA_eq_relation}
  r_0 &= a_0 = 0 \nonumber \\
  r_i &= \left( \int_{0}^{a_i} x \, p(x) dx \right) / P(a_i) & (i > 0)
\end{align}
\end{theorem}

\begin{proof}
By equation \ref{eq:MVA_eq_relation}, we have $r_i P(a_i) = \int_{0}^{a_i}
x\,p(x)dx$ for all $i$ (TOO BAD this won't hold if $a_0 > 0$). Thus the right
hand side of equation \ref{eq:MVA_eq} is:
\begin{align*}
	& \int_{a_{i-1}}^{a_i} a_i p(x) dx - \int_{a_{i-1}}^{a_i} x \, p(x) dx + P(a_{i-1})(a_i-r_{i-1}) \\
	&= a_i P(a_i) - \cancel{a_i P(a_{i-1})} - r_i P(a_i) + \cancel{r_{i-1} P(a_{i-1})} \\
		& ~~~ + \cancel{P(a_{i-1}) a_i} - \cancel{P(a_{i-1}) r_{i-1}}\\
	&= \mbox{left hand side of equation\ref{eq:MVA_eq}}
\end{align*}
\end{proof}

This tells us that a bidder will bid in a round of MVA if and only if the
expected second highest bidding conditional on this bidder's valuation is the
highest is greater than the reserve price of that round. For example, if the
distribution is uniform, i.e. $P(x) = x^{n-1}$, $r_i = \frac{n-1}{n} a_i$ for
$i > 0$.



\subsection{Preliminary Revenue Comparisons}

\section{Optimal Mechanisms with Efficiency and Only Seller's Cost}

In this section, we consider a simplified optimizing problem with efficiency
constraint and the cost is only charged to the seller.  Though these two
constraints simplify our problem a lot, they are very common in real cases such
as craigslist or moving sales in mailing-lists.  We will first prove that MVAs
are optimal and then we'll try to find the specific MVA that achieves the
optimality.

\subsection{Optimality of MVAs}

\begin{theorem}\label{theorem1}

Among all mechanisms that can include multiple rounds of broadcasts and are
required to be efficient (allocate the item to the bidder with highest
valuation), Multi-round Vickrey Auctions (MVAs) are of minimum cost.

\end{theorem}


\begin{corollary}

If all broadcast costs and bidding costs are charged to sellers, MVAs are
optimal if efficiency is required.  Such optimal MVA is the one that minimizes
the overall cost.

\end{corollary}

\subsection{Cost Minimized $\alpha$-MVA}

\subsection{Approximation of $\alpha$ and Experiments}

It's difficult to get an exact closed formula for optimal $\alpha$. Thus we are
going to use a closed formula to approximate this $\alpha$. We'll conduct
experiments to compare our approximation with the optimal $\alpha$ that's
computed numerically. We are also going to show comparisons between optimal
MVA, approximate optimal MVA and other conventional mechanisms such as Vickrey
auctions.

\section{Optimal Mechanisms with Both Seller's and Bidder's Cost}

In this section, we take out the constraint and prove that MVAs are optimal in
general. We will also try to find the specific MVA to achieve such optimality
and it turns out to be significantly more complicated than our previous
simplified case.

\subsection{Spending Equivalence Theorem and Revenue Optimization Strategy}

\begin{definition}

We say mechanisms satisfy relaxed efficiency constraint with lowest type $l$
if:

    \begin{enumerate}

    \item They only allocate the item to bidders whose valuation are at least
    $l$ (the lowest type is $l$)

    \item If they will allocate the item, they will always allocate the item to
    the bidder with highest valuation.

    \end{enumerate}

When we say a mechanism with a lowest type $l$, we imply that this mechanism
satisfy relatex efficiency constraint with lowest $l$.

\end{definition}

\begin{theorem}

For all mechanisms with the same lowest type $l$, they will have the same
overall spending from all bidders (including bidders' bidding costs and
payments to sellers).

\end{theorem}

\begin{corollary}

For mechanisms with a fixed lowest type $l$, the maximum utility for sellers is
achieved when the mechanism minizes the cost.

\end{corollary}

\begin{theorem}

MVAs have the minimum cost among all mechanisms with a lowest type $l$

\end{theorem}

\begin{theorem}

MVAs are optimal. (watch out for relaxed efficiency constraint)

\end{theorem}

\begin{corollary}

Shall we increase lowest type $l$ a little above Myerson's optimal lowest type
to trade payment with cost? Or is it optimal to use Myerson's $l$ and then
minimize cost according to that?

\end{corollary}

\subsection{Experiments to Discover Optimal MVA with a Given Lowest Type}

\subsection{Analysis of Optimal MVA with Lowest Type}

\subsection{Using Piecewise Linear MVA to Approximate Optimal MVA}

\subsection{Choosing Lowest Type?}

\subsection{Experiments}

Now we are going to compare revenue in general cases. We not only compare our
approximate optimal MVAs to optimal MVAs computed numerically, but also compare
MVAs to other conventional mechanisms.

\section{Conclusion}


\end{document}
