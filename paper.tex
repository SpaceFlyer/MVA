\documentclass{aamas2012}

\pdfpagewidth=8.5truein
\pdfpageheight=11truein

\begin{document}

\title{Optimal Broadcast Auctions with Costly Actions}

\numberofauthors{2}

\author{
\alignauthor
XXX
\alignauthor
YYY
}


\maketitle

\section{Introduction}

\section{Cost Model and Settings}
\newtheorem{definition}{Definition}
\begin{definition}
In our settings, there's only one seller selling one item to $n$ i.i.d. bidders
whose valuation distribution (CDF) is continuous over $[0, 1]$.
The seller can broadcast a message to all bidders with a 
cost $b$ (the broadcast cost). A bidder may remain silent after that broadcast with no cost.
If not, it costs $c_0$ if the bidder's bidding(reply) doesn't
imply positive valuation. Otherwise, if the bidding implies a positive 
valuation, a greater bidding cost $c > c_0$ is charged.
Such bidding cost $c$ main contain two parts, $c_1, c_2~~(c_1+c_2 = c)$. 
The first part $c_1$ is charged to the seller while the second is charged to
the corresponding bidder.
\end{definition}

\section{Optimal Mechanisms with Efficiency and Only Seller's Cost}

\subsection{Optimality of Multi-round Vickrey Auctions (MVA)}

\newtheorem{theorem}{Theorem}
\begin{theorem}\label{theorem1}
Suppose there's one seller selling one item to $n$ i.i.d. bidders
whose valuation distribution (CDF) is continuous over $[0, 1]$.
The seller can broadcast a message to all bidders with a 
cost $b$. A bidder may remain silent after that broadcast with no cost.
If not, it costs $c_0$ if the bidder's bidding(reply) doesn't
imply positive valuation. Otherwise, if the bidding implies a positive 
valuation, a greater bidding cost $c_1 > c_0$ is charged.
Among all mechanisms that can include multiple rounds of broadcasts
and are required to be efficient (allocate the item to the bidder with
highest valuation), Multi-round Vickrey Auctions (MVAs) are of minimum
cost.
\end{theorem}

\newtheorem{corollary}{Corollary}
\begin{corollary}
If all broadcast costs and bidding costs are charged to sellers,
MVA gives the maximum utility to sellers if efficiency is required.
Such optimal MVA is the one that minimizes the overall cost.
\end{corollary}

\subsection{Cost Minimized $\alpha$-MVA}

\subsection{Approximation of $\alpha$ and Experiments}

\section{Optimal Mechanisms with Both Seller's and Bidder's Cost}

\subsection{Payment Equivalence Theorem and Revenue Optimization Strategy}

\begin{theorem}
If two mechanisms satisfy:
\begin{enumerate}
\item They only allocate the item to bidders whose valuation
are at least $l$ (the lowest type is $l$)
\item If they will allocate the item, they
will always allocate the item to the bidder with highest valuation.
\end{enumerate}
They will have the same overall payment from all bidders 
(including bidders' bidding costs and payments to sellers).
\end{theorem}

\begin{corollary}
For mechanisms that have a fixed lowest type $l$ and
will always allocate the item to the bidder with highest valuation if it's at least $l$, 
the maximum utility for sellers is achieved
when the mechanism minize the cost.
\end{corollary}

\begin{theorem}
%Suppose there's one seller selling one item to $n$ i.i.d. bidders
%whose valuation distribution CDF is continuous over $[0, 1]$.
%The seller can broadcast a message to all bidders with a 
%cost $b$. A bidder may remain silent after that broadcast with no cost.
%If not, it costs $c_0$ if the bidder's bidding(reply) doesn't
%imply positive valuation. Otherwise, if the bidding implies a positive 
%valuation, two bidding costs $c_1, c_2$ where $c_1+c_2 > c_0$ are charged
%to the seller and corresponding bidder respectively.
%Among all mechanisms that can include multiple rounds of broadcasts
%and are required to allocate the item to the bidder with highest valuation
%not less than lowest type $l$ (no allocation is no such bidder exists), 
%Multi-round Vickrey Auctions (MVA) are of minimum overall cost.

Under the same circumstances of theorem \ref{theorem1}, except that
the efficiency constraint is replaced by lowest type $l$ and the bidding cost is 
now also charged to corresponding bidder,
MVAs are of the minimum cost among all those mechanisms that have a fixed 
lowest type $l$ and will always allocate the item to the bidder 
with highest valuation if it's at least $l$.
\end{theorem}

\begin{corollary}
MVAs are optimal in terms of maximizing seller's utility.
\end{corollary}

\begin{corollary}
Shall we increase lowest type $l$ a little above Myerson's
optimal lowest type to trade payment with cost? Or is it optimal
to use Myerson's $l$ and then minimize cost according to that?
\end{corollary}

\subsection{Experiments to Discover Optimal MVA with a Given Lowest Type}

\subsection{Analysis of Optimal MVA with Lowest Type}

\subsection{Using Piecewise Linear MVA to Approximate Optimal MVA}

\subsection{Choosing Lowest Type?}

\section{Conclusion}


\end{document}
