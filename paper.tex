\documentclass{aamas2012}

\usepackage{cancel}
\usepackage{subfigure}

\pdfpagewidth=8.5truein \pdfpageheight=11truein

\newtheorem{theorem}{Theorem}
\newtheorem{definition}{Definition}
\newtheorem{corollary}{Corollary}
\newtheorem{lemma}{Lemma}

\begin{document}

\title{Optimal Broadcast Auctions with Costly Actions}

\numberofauthors{2}

\author{ \alignauthor XXX \alignauthor YYY }


\maketitle

\section{Introduction}

\section{Cost Model and Settings}
\label{sec:model}

In this section, we formally define our cost model and explain its motivation. 
%Our
%cost model is inspired from online second-hand item transactions, including
%examples like eBay, craigslist and universities' mailing-lists.


\begin{definition}[Setting]\label{def:model}
One seller is selling one item to $n$ buyers (bidders)
whose valuations $v_i~(1 \leq i \leq n)$ are independently and identically
distributed (i.i.d.) over $[0, 1)$ with PDF $f(x)$ and CDF $F(x)$. The seller
can broadcast a message to all bidders, at cost $b$ to
the seller.  A bidder can reply to that broadcast, or remain silent.
If a bidder replies (does not stay silent), this comes at a cost $\beta_1$ to the
seller and a cost $\beta_2$ to that bidder, for a total cost of $c=\beta_1+\beta_2$.
%vc: is this necessary?  discuss where it's needed.
%We restrict our attention to pure strategies, so that bidders reply
%deterministically to queries.
%The bidder's reply is deterministic with respect to the seller's
%broadcast (for simplicity, we only consider pure strategy equilibrium).
\end{definition}

%The major difference of this model are cost and broadcast capability. We will
%explain bidding cost $c$, broadcast capability and broadcast cost $b$ in
%the following text.
%
%The bidding costs $c$ may be caused by communication and other verification
%required to bid. For example, the bidder may have to input his credit card
%number and prepay an amount of money. Without such verification, a bidder may
%bid very high and refuse to pay in the end. A verification for the seller might
%also be needed. For example, a bidder may want to set up an appointment with
%the seller to check the item.  Setting up such an appointment might be costly
%because they need to dicuss time and place via emails or phone calls.
%Attending that appointment may also cost travel fees and time.  Note that our
%bidding cost is different from conventional participation cost studied before
%\cite{Stegeman95:ParticipationCost, Tan2006:EquilibriaParticipationCost}. In
%our model, it is free for bidders to participate without bidding, which is more
%close to most Internet platforms. In another word, bidders do not have to buy a
%ticket to walk in and observe.

Two key aspects of this model are that (1) staying silent comes at no cost
and (2) replying comes at a positive cost, {\em and this positive cost is
  the same no matter how complex the query and answer are}.  This is
motivated by the settings discussed in the introduction, where a bidder can
observe the process of the auction (or messages posted on a board) silently
at no cost, but once the bidder acts in the auction, costs occur---e.g., the
bidder has to submit credit card information, the bidder and the seller
have to arrange an in-person meeting, etc.).  A key aspect of such costs is
that they tend to be the same regardless of the level of detail in the
bidder's answer: for example, if the bidder just reports having a valuation
greater than \$10 without specifying what it is exactly (rather than
reporting a valuation of exactly \$14), this is not likely to reduce any of
the above costs.  In particular, the seller is likely to want to verify the
bidder's authenticity at any point where the bidder's reply affects the
course of the auction from then on.  
%Indeed, in the auction mechanisms that
%we propose, whenever a bidder replies, that bidder will reveal her valuation exactly.
This leads to the following easy proposition:

\begin{proposition}
\label{prop:fullrevelation}
  In the model defined in~\ref{def:model}, without loss of optimality, we
  can restrict our attention to (broadcast) queries that result in each
  bidder either staying silent, or immediately revealing his exact
  valuation to the seller.
\end{proposition}

Another notable point is that we restrict communication from the seller to
broadcast queries.  This is a common restriction: any sealed-bid auction
can be considered a broadcast auction with only one broadcast: the reserve
price.  The bisection auction~\cite{Herings2009:BisectionAuction} is an
example auction with many rounds of broadcasts. In each round, it
broadcasts a price and asks bidders to reply whether their valuation is
above or below that price.  Besides the broadcast model being simple,
natural, and common in existing auction mechanisms, it is also naturally
motivated in the Internet domains we consider: the bidders are entirely
anonymous until their first reply, so before this point querying such
bidders individually is not feasible and they can only be reached by, say,
posting on a public website; and after they have replied, we will know
their valuation exactly (by Proposition~\ref{prop:fullrevelation}) and we no
longer need to query them.  Of course, there are offline cases where the
set of bidders is small and explicit (e.g., the government wants to sell
land or spectrum to one of three known companies); in such settings, it can
indeed be helpful for the seller to communicate with bidders individually
\cite{McAfee88:SearchMechanisms,Sandholm06:Sequences}.  Such settings do
not fit our model; we explicitly focus on highly anonymous settings, and
the costs that the seller incurs from the broadcast query correspond to the
time, effort, and third-party charges associated with posting a public
message.


%We introduce broadcast capability because it is shared by many auctions.  For
%example, a Vickrey auction or a first-price auction with reserve price can both be
%described as a broadcast auction with only one broadcast: telling every bidder
%the reserve price. The bisection auction \cite{Herings2009:BisectionAuction} is an
%example with many rounds of broadcasts. In each round , it will broadcast
%a price and ask bidders to reply whether their valuation is beyond or below that
%price.  In real world, sellers make such broadcasts via sending emails to a
%bunch of receivers (typically a mailing-list), listing items on a platform such
%as craigslist and eBay, or even showing ads on Internet/TV. Such broadcast
%activities costs either money (e.g. a list or ads fee) or time and effort (e.g.
%writing and sending emails).

%Note that in our model, we only give sellers broadcast capability so they
%cannot find or communicate with each bidder one by one. The first reason to
%have this constraint is there are too many potential bidders on the Internet
%where our model is originated so it is hard to explicitly find
%them one by one. By contrast, in offline cases where the set of bidders
%are small and explicit (e.g. the government want to sell a land to one of three
%companies), it might be helpful to let the seller communicate with bidders one
%by one \cite{McAfee88:SearchMechanisms}. The second reason to have this constraint is
%that we want to focus on mechanisms that avoids time consuming bargaining
%between two individuals. It makes the mechanism simple, fair and convenient.

%Finally, we define optimal mechanisms to be the ones that maximze seller's
%profit since when facing many different auction mechanisms, a rational seller
%will choose the one that gives him maximum profit.



\begin{definition}
%A {\em (broadcast) query} consists of a subset $Q \subseteq [0,1)$, where
%agents with a valuation in $Q$  respond with their exact valuation and 
%agents with a valuation in $[0,1) \ Q$ stay silent.
%vc: do we need determinism???
  A {\em mechanism} in our setting consists of (1) a full contingency plan
  for which query to broadcast at each point, depending on answers given so
  far, and a termination condition; and (2) an allocation and pricing rule
  that is defined on each terminal state.  A mechanism is {\em individually
    rational} if losing bidders never pay and winning bidders never pay
  more than their valuations.  We say an individually rational mechanism is
  {\em optimal} if it has a Bayes-Nash equilibrium for the bidders that maximizes the
  seller's profit (among all Bayes-Nash equilibria of all individually
  rational auction mechanisms).  Here, seller profit is revenue minus
  seller elicitation costs.
%gives the seller maximum profit
%which equals to all the value payed to this seller (revenue) minus all the cost charged
%to this seller. 
  A class of mechanisms is optimal if it contains at least one optimal
  mechanism.
\end{definition}


\section{Optimal Mechanisms with Efficiency and Only Seller's Cost}

In this section, we consider a simplified optimizing problem with efficiency
constraint and the cost is only charged to the seller.  Though these two
constraints simplify our problem a lot, they are very common in real cases such
as craigslist or moving sales in mailing-lists. (more to explain efficiency
and no bidder's cost).

First of all, we introduce a mechanism called Multi-round Vickrey Auction (MVA)
based on what's been using in realworld online second-hand item transactions.
Then we prove that MVAs are optimal. After that we'll try to find the specific
MVA that achieves the optimality. Finally, we conduct some experiments to
compare the optimal MVA with other mechanisms.

\subsection{Multi-round Vickrey Auctions}

A Multi-round Vickrey Auction (MVA) has multiple rounds of Vickrey
auctions with progressively decreasing reserve prices. This kind of auction
effectively occurs on eBay. The seller may set up a reserve price and let
buyers bid for this item. The proxy bidding functionality makes such an auction
equivalent to a Vickrey auction with a reserve price. If no buyers bid for a
given reserve price, the seller may lower the reserve price, which makes the
whole process equivalent to an MVA.

\begin{definition}(Multi-round Vickrey Auction, MVA)

In a Multi-round Vickrey Auction (MVA), there's a sequence of reserve prices
$r_k, r_{k-1}, \ldots, r_0$ where $r_k > r_{k-1}$. The seller creates a Vickrey
auction with a reserve price $r_{k-i}$ at time $i$ (or round $k-i$). In each
Vickrey auction, if only one buyer bids, he/she gets the item and pays reserve
price. Otherwise, the buyer with the highest bidding gets the item and pays the
second highest bidding.

\end{definition}

MVAs require Vickrey auctions (or equivalent English auctions) as basic steps.
In reality, however, such functionality won't always be provided by online
platforms such as craigslist. Thus a simplified version of MVA occur very often
in those platforms. People call it first-come first served which means for
every reserve price $r_i$, the first one who accept that price wins the item
and pays $r_i$ directly. This mechanism may loose revenue and social efficiency
as the person with lower valuation $p$ may get the item for $r_i$ while there's
someone else who is willing to pay a higher amount of $q$ where $r_i \leq p < q
< r_{i+1}$. We won't focus on this first-come first served mechanism because
it's harder to analyze analytically and it's inferior than MVAs in terms of
both sellers' utility and social welfare.

Since there's no cost charged to buyers, it's obvious to see that whenever a
bidder decides to bid, he/she must bid truthfully. Thus the Bayesian Nash
Equilibria (BNE) for MVAs can be described as $k$ thresholds $a_k, a_{k-1},
\ldots, a_0$ where $a_i > a_{i-1}$. Whenever a bidder's valuation for the item
is greater than $a_i$, he/she is going to bid in round $i$ whose reserve price
is $r_i$. Because of efficiency constraint we also have $r_0 = a_0 = 0$.

In later analysis, we will think about the equilibrium from another
perspective.  We firstly decide thresholds $a_i$ since it's more meaningful for
bidders to make decisions and for us to make analysis. For a set of thresholds,
we then determine the right reserve prices $r_i$ so bidders are incentive
compatible to bid according to $a_i$. The following equations connects $a_i$
and $r_i$:
\begin{align}\label{eq:MVA_eq}
    &r_0 = a_0 = 0 \mbox{ and }
    \forall i ~(1 \leq i \leq k),\nonumber\\
    &~~P(a_{i})(a_{i}-r_i) =
    \int_{a_{i-1}}^{a_{i}}(a_{i}-x)p(x)dx+P(a_{i-1})(a_{i}-r_{i-1})
\end{align}
assuming
\begin{align*}
    P(x) &:= F(x)^{n-1}\\
    p(x) &:= P'(x) = (n-1)F(x)^{n-2}
\end{align*}
The equation \ref{eq:MVA_eq} says that the bidder with valuation $a_i$ should be
indifferent from bidding in round $i$ (the left hand side) and bidding in round
$i-1$(the right hand side).  The following theorem describes the equilibrium of
MVAs determined by equations above.

\begin{theorem}
In a MVA with a pure strategy Bayesian Nash Equibibrium characterized by
thresholds $a_k, a_{k-1}, \ldots, a_0$ where the bidder with valuation greater
than $a_i$ (but not greater than $a_{i+1}$) will bid in round $i$ (recall that
round $k$ is the first round and round $0$ is the last round), the reserve
prices $r_i$ shall be:
\begin{align}\label{eq:MVA_eq_relation}
  r_0 &= a_0 = 0 \nonumber \\
  r_i &= \left( \int_{0}^{a_i} x \, p(x) dx \right) / P(a_i) & (i > 0)
\end{align}
\end{theorem}

\begin{proof}
By equation \ref{eq:MVA_eq_relation}, we have $r_i P(a_i) = \int_{0}^{a_i}
x\,p(x)dx$ for all $i$. Thus the right
hand side of equation \ref{eq:MVA_eq} is:
\begin{align*}
	& \int_{a_{i-1}}^{a_i} a_i p(x) dx - \int_{a_{i-1}}^{a_i} x \, p(x) dx + P(a_{i-1})(a_i-r_{i-1}) \\
	&= a_i P(a_i) - \cancel{a_i P(a_{i-1})} - r_i P(a_i) + \cancel{r_{i-1} P(a_{i-1})} \\
		& ~~~ + \cancel{P(a_{i-1}) a_i} - \cancel{P(a_{i-1}) r_{i-1}}\\
	&= \mbox{left hand side of equation\ref{eq:MVA_eq}}
\end{align*}
\end{proof}

This tells us that a bidder will bid in a round of MVA if and only if the
expected second highest bidding conditional on this bidder's valuation is the
highest is greater than the reserve price of that round. For example, if the
distribution is uniform, i.e. $F(x) = x$, $r_i = \frac{n-1}{n} a_i$ for
$i > 0$.

\subsection{Optimality of MVAs}

\begin{theorem}\label{theorem1}

Among all mechanisms that can include multiple rounds of broadcasts and are
required to be efficient (allocate the item to the bidder with highest
valuation), Multi-round Vickrey Auctions (MVAs) are of minimum cost.

\end{theorem}


\begin{corollary}

If all broadcast costs and bidding costs are charged to sellers, MVAs are
optimal if efficiency is required.  Such optimal MVA is the one that minimizes
the overall cost.

\end{corollary}

\subsection{Cost Minimized $\alpha$-MVA}

\subsection{Approximation of $\alpha$ and Experiments}

It's difficult to get an exact closed formula for optimal $\alpha$. Thus we are
going to use a closed formula to approximate this $\alpha$. We'll conduct
experiments to compare our approximation with the optimal $\alpha$ that's
computed numerically. We are also going to show comparisons between optimal
MVA, approximate optimal MVA and other conventional mechanisms such as Vickrey
auctions.




\section{Optimal Mechanisms with Both Seller's and Bidder's Cost}

In this section, we take out the constraint and prove that MVAs are optimal in
general. We will also try to find the specific MVA to achieve such optimality
and it turns out to be significantly more complicated than our previous
simplified case.

\subsection{Spending Equivalence Theorem and Revenue Optimization Strategy}

\begin{definition}

We say mechanisms satisfy relaxed efficiency constraint with lowest type $l$
if:

    \begin{enumerate}

    \item They only allocate the item to bidders whose valuation are at least
    $l$ (the lowest type is $l$)

    \item If they will allocate the item, they will always allocate the item to
    the bidder with highest valuation.

    \end{enumerate}

When we say a mechanism with a lowest type $l$, we imply that this mechanism
satisfy relatex efficiency constraint with lowest $l$.

\end{definition}

\begin{theorem}

For all mechanisms with the same lowest type $l$, they will have the same
overall spending from all bidders (including bidders' bidding costs and
payments to sellers).

\end{theorem}

\begin{corollary}

For mechanisms with a fixed lowest type $l$, the maximum utility for sellers is
achieved when the mechanism minizes the cost.

\end{corollary}

\begin{theorem}

MVAs have the minimum cost among all mechanisms with a lowest type $l$

\end{theorem}

\begin{theorem}

MVAs are optimal. (watch out for relaxed efficiency constraint)

\end{theorem}

\begin{corollary}

Shall we increase lowest type $l$ a little above Myerson's optimal lowest type
to trade payment with cost? Or is it optimal to use Myerson's $l$ and then
minimize cost according to that?

\end{corollary}

\subsection{Experiments to Discover Optimal MVA with a Given Lowest Type}

\subsection{Analysis of Optimal MVA with Lowest Type}

\subsection{Using Piecewise Linear MVA to Approximate Optimal MVA}

\subsection{Choosing Lowest Type?}

\subsection{Experiments}

Now we are going to compare revenue in general cases. We not only compare our
approximate optimal MVAs to optimal MVAs computed numerically, but also compare
MVAs to other conventional mechanisms.

\section{Conclusion}


\end{document}
