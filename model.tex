\section{Cost Model and Settings}
\label{sec:model}

In this section, we formally define our cost model and explain its motivation. 
%Our
%cost model is inspired from online second-hand item transactions, including
%examples like eBay, craigslist and universities' mailing-lists.


\begin{definition}[Setting]\label{def:model}
One seller is selling one item to $n$ buyers (bidders)
whose valuations $v_i~(1 \leq i \leq n)$ are independently and identically
distributed (i.i.d.) over $[0, 1)$ with PDF $f(x)$ and CDF $F(x)$
which are assumed to have no gaps ($F(x)$ is strictly increasing
everywhere).
The seller
can broadcast a message to all bidders, at cost $b$ to
the seller.  A bidder can reply to that broadcast, or remain silent.
If a bidder replies (does not stay silent), this comes at a cost $\beta_1$ to the
seller and a cost $\beta_2$ to that bidder, for a total cost of $c=\beta_1+\beta_2$.
%vc: is this necessary?  discuss where it's needed.
%We restrict our attention to pure strategies, so that bidders reply
%deterministically to queries.
%The bidder's reply is deterministic with respect to the seller's
%broadcast (for simplicity, we only consider pure strategy equilibrium).
\end{definition}

%The major difference of this model are cost and broadcast capability. We will
%explain bid cost $c$, broadcast capability and broadcast cost $b$ in
%the following text.
%
%The bid costs $c$ may be caused by communication and other verification
%required to bid. For example, the bidder may have to input his credit card
%number and prepay an amount of money. Without such verification, a bidder may
%bid very high and refuse to pay in the end. A verification for the seller might
%also be needed. For example, a bidder may want to set up an appointment with
%the seller to check the item.  Setting up such an appointment might be costly
%because they need to dicuss time and place via emails or phone calls.
%Attending that appointment may also cost travel fees and time.  Note that our
%bid cost is different from conventional participation cost studied before
%\cite{Stegeman95:ParticipationCost, Tan2006:EquilibriaParticipationCost}. In
%our model, it is free for bidders to participate without bidding, which is more
%close to most Internet platforms. In another word, bidders do not have to buy a
%ticket to walk in and observe.

Two key aspects of this model are that (1) staying silent comes at no cost
and (2) replying comes at a positive cost, {\em and this positive cost is
  the same no matter how complex the query and answer are}.  This is
motivated by the settings discussed in the introduction, where a bidder can
observe the process of the auction (or messages posted on a board) silently
at no cost, but once the bidder acts in the auction, costs occur---e.g., the
bidder has to submit credit card information, the bidder and the seller
have to arrange an in-person meeting, etc.).  A key aspect of such costs is
that they tend to be the same regardless of the level of detail in the
bidder's answer: for example, if the bidder just reports having a valuation
greater than \$10 without specifying what it is exactly (rather than
reporting a valuation of exactly \$14), this is not likely to reduce any of
the above costs.  In particular, the seller is likely to want to verify the
bidder's authenticity at any point where the bidder's reply affects the
course of the auction from then on.  
%Indeed, in the auction mechanisms that
%we propose, whenever a bidder replies, that bidder will reveal her valuation exactly.
This leads to the following easy proposition:

\begin{proposition}
\label{prop:sufficient}
  In the model defined in~\ref{def:model}, without loss of optimality, we
  can restrict our attention to (broadcast) queries that result in each
  bidder either staying silent, or immediately revealing his exact
  valuation to the seller.
\end{proposition}

Another notable point is that we restrict communication from the seller to
broadcast queries.  This is a common restriction: any sealed-bid auction
can be considered a broadcast auction with only one broadcast: the reserve
price.  The bisection auction~\cite{Herings2009:BisectionAuction} is an
example auction with many rounds of broadcasts. In each round, it
broadcasts a price and asks bidders to reply whether their valuation is
above or below that price.  Besides the broadcast model being simple,
natural, and common in existing auction mechanisms, it is also naturally
motivated in the Internet domains we consider: the bidders are entirely
anonymous until their first reply, so before this point querying such
bidders individually is not feasible and they can only be reached by, say,
posting on a public website; and after they have replied, we will know
their valuation exactly (by Proposition~\ref{prop:sufficient}) and we no
longer need to query them.  Of course, there are offline cases where the
set of bidders is small and explicit (e.g., the government wants to sell
land or spectrum to one of three known companies); in such settings, it can
indeed be helpful for the seller to communicate with bidders individually
\cite{McAfee88:SearchMechanisms,Sandholm06:Sequences}.  Such settings do
not fit our model; we explicitly focus on highly anonymous settings, and
the costs that the seller incurs from the broadcast query correspond to the
time, effort, and third-party charges associated with posting a public
message.


%We introduce broadcast capability because it is shared by many auctions.  For
%example, a Vickrey auction or a first-price auction with reserve price can both be
%described as a broadcast auction with only one broadcast: telling every bidder
%the reserve price. The bisection auction \cite{Herings2009:BisectionAuction} is an
%example with many rounds of broadcasts. In each round , it will broadcast
%a price and ask bidders to reply whether their valuation is beyond or below that
%price.  In real world, sellers make such broadcasts via sending emails to a
%bunch of receivers (typically a mailing-list), listing items on a platform such
%as craigslist and eBay, or even showing ads on Internet/TV. Such broadcast
%activities costs either money (e.g. a list or ads fee) or time and effort (e.g.
%writing and sending emails).

%Note that in our model, we only give sellers broadcast capability so they
%cannot find or communicate with each bidder one by one. The first reason to
%have this constraint is there are too many potential bidders on the Internet
%where our model is originated so it is hard to explicitly find
%them one by one. By contrast, in offline cases where the set of bidders
%are small and explicit (e.g. the government want to sell a land to one of three
%companies), it might be helpful to let the seller communicate with bidders one
%by one \cite{McAfee88:SearchMechanisms}. The second reason to have this constraint is
%that we want to focus on mechanisms that avoids time consuming bargaining
%between two individuals. It makes the mechanism simple, fair and convenient.

%Finally, we define optimal mechanisms to be the ones that maximze seller's
%profit since when facing many different auction mechanisms, a rational seller
%will choose the one that gives him maximum profit.

%The following restriction is natural in our context:

\begin{definition}[No blind allocation]\label{def:allocation_cost}
An item cannot be awarded to a bidder who has remained silent to all queries.
%A mechanism with allocation rule $$p: (v_1, v_2, \ldots, v_n) \rightarrow
%(p_1, p_2, \ldots, p_n)$$ does not allocate blindly if it satisfies: if $p_i >
%0$ for some valuation profile $(v_1, v_2, \ldots, v_n)$, there must be a
%broadcast query that the $i$-th bidder ($v_i$) reply to the seller
%under that profile setting.
\end{definition}



\begin{definition}
%A {\em (broadcast) query} consists of a subset $Q \subseteq [0,1)$, where
%agents with a valuation in $Q$  respond with their exact valuation and 
%agents with a valuation in $[0,1) \ Q$ stay silent.
%vc: do we need determinism???
  A {\em mechanism} in our setting consists of (1) a full contingency plan
  for which query to broadcast at each point, depending on answers given so
  far, and a termination condition; and (2) an allocation and pricing rule
  that is defined on each terminal state, satisfying no blind allocation.  A mechanism is {\em individually
    rational} if losing bidders never pay and winning bidders never pay
  more than their valuations.  We say an individually rational mechanism is
  {\em optimal} if it has a Bayes-Nash equilibrium for the bidders that maximizes the
  seller's profit (among all Bayes-Nash equilibria of all individually
  rational auction mechanisms).  Here, seller profit is revenue minus
  seller's costs.
%gives the seller maximum profit
%which equals to all the value payed to this seller (revenue) minus all the cost charged
%to this seller. 
  A class of mechanisms is optimal if it contains at least one optimal
  mechanism.
\end{definition}
