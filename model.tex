\section{Cost Model and Settings}

In this section, we define our cost model and explain why it makes sense. Our
cost model is inspired from online second-hand item transactions, including
examples like eBay, craigslist and universities' mailing-lists.


\begin{definition}\label{def:model}

In our settings, there's one seller selling one item to $n$ buyers (bidders)
whose valuations $v_i~(1 \leq i \leq n)$ are independently and identically
distributed (i.i.d.) over $[0, 1]$ with PDF $f(x)$ and CDF $F(x)$. The seller
can broadcast a message to all bidders which costs $b$ (the broadcast cost) for
the seller.  A bidder may reply to that broadcast with cost $c$ (bidding cost)
or remain silent with no cost. Such bidding cost $c$ may contain two parts,
$\beta_1, \beta_2$ where $\beta_1+\beta_2 = c$.  The first part $\beta_1$ is
charged to the seller while the second is charged to the corresponding bidder.
The bidder's reply should be deterministic with respect to the seller's
broadcast (for simplicity, we only consider pure strategy equilibrium).

\end{definition}

Most settings in definition \ref{def:model} are quite standard except for the
cost and broadcast capability.

The bidding costs $c$ may be caused by communication and other verification
actions required to put a bidding. For example, the bidder may have to input
his credit card number and prepay an amount of money. Without such verification
for the bidder, a bidder may bid very high and refuse to pay in the end. A
verification for the seller might also be needed. For example, a bidder may
want to set up an appointment with the seller to check the item.  Setting up
such an appointment might be costly because they need to dicuss time and place
via emails or phone calls.  Attending that appointment may also cost travel
fees and time. 

We introduce broadcast capability because it's shared by many auctions.  For
example, a Vickrey auction or a first-price auction with reserve price can be
described as a broadcast auction with only one broadcast: telling every bidder
the reserve price. The bisection auction [cite bisection auctions] is another
example which has many rounds of broadcasts. In each round , it will broadcast
a price and ask bidders to reply whether his valuation is beyond or below that
price.  In real world, sellers make such broadcasts via sending emails to a
bunch of receivers (typically a mailing-list), listing items on a platform such
as craigslist and eBay, or even showing ads on Internet/TV. Such broadcast
activities costs either money (e.g. a list or ads fee) or time and effort (e.g.
writing and sending an email).

Note that in our model, we only give sellers broadcast capability so they
cannot find or communicate with each bidder one by one. The first reason to
have this constraint is there are too many potential bidders on the Internet
(our model focus on online item transactions) and it's hard to explicitly find
them one by one. On the other hand, in offline cases where the set of bidders
are small and explicit (e.g. the government want to sell a land to one of three
companies), it might be helpful to let the seller communicate with bidders one
by one [cite search mechanisms]. The second reason to have this constraint is
that we want to focus on mechanisms that avoids time consuming bargaining. Such
feature is very important as one of the most vital advantages of online
transactions are their convenience and the time consuming bargainning can ruin
it.

Finally, we define optimal mechanisms to be the ones that maximze seller's
utility since when facing many different auction mechanisms, a rational seller
will choose the one that gives him maximum utility.

\begin{definition}

We say a mechanism is optimal if it gives the seller maximum utility (revenue)
which equals to all the value payed to this seller minus all the cost charged
to this seller. A class of mechanisms are optimal if they contain one optimal
mechanism.

\end{definition}
