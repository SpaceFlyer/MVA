\section{Optimal Mechanisms with Both Seller's and Bidder's Cost}

In this section, we take out the constraints and prove that MVAs are optimal in
general. We will also try to find the specific MVA to achieve such optimality,
which turns out to be significantly more complex than previous
simplified case.

The first constraint we are going to remove is seller's cost only.  It's
exciting to introduce bidder's bidding cost since it occurs very often in real
cases and it plays an important role.  Sending emails, making phone calls,
enterring credit card nubmers, depositing money and clicking buttons are all
costly for bidders, though sometimes very tiny.  Bidders may not bid when this
cost is greater than their expected utility. Note that even if the valuation is
very high, the expected utility can be very small because of tense competition,
which is very common in online cases as $n$, the number of potential bidders,
is very large. 

This behaviour (bidders won't bid because of competitions) is very different
compared to that in previous model [cite Sequential Optimal Auctions] of
sequential auctions. In that model, there's a time discount which
makes bidders eager to bid in early rounds with high reserve prices to avoid
waiting lost. That makes a lot sense in some cases but sometimes it may not.
For example, if the seller posts an auction with a very low reserve price in
the first round, most bidders with high valuation must be happy to bid
according to that time discount cost model. But this may not be true. For
example, when I encounter such an auction online \footnote{For example, when I
see a very good item in the Auction House of Diablo III with a very low current
bidding}, I might be very reluctant to bid because there's a big probability
that my bidding will be over taken by someone else's so it's just a waste of
effort. Our cost model can describe this behaviour very well.

Assume an extreme case where the broadcast cost is $0$, the bidder's bidding
cost is $0.1$ and there are $n \rightarrow \infty$ many $[0, 1]$-uniform
distributed bidders. In a Dutch auction (a Dutch auction has infinite many
rounds of broadcasts so we have to set braodcast cost to $0$), only one bidder
is expected to bid (no competition), thus every bidder with a valuation $v_i
> 0.1$ should be benifitable to bid when the reserve price drops to a little
> bit below
$v_i-0.1$ (recall that $n \rightarrow \infty$).  In a Vickrey auction, however,
the competition is very tense. Only bidders with valuation greater than $t$ can
accept such intense competition where $t$ satisfies $t^{n-1}t - 0.1 = 0$ (the
expected utility for a bidder with valuation $t$ is $0$).  Thus $t =
\sqrt[n]{0.1}$ which is arbitrary close to $1$ as $n$ grows to infinity.  Thus
almost all bidders can't bare this competition when $n$ is really large.

So a bad mechanism (e.g. a Vickrey auction) with too much cost will have less
participations and therefore decreases seller's utility significantly.  To see
this, look at previous extrame case again.  The revenue of Dutch auction will
converge to $0.9$ (someone with valuation very close to $1$ will bid for price
very close to $0.9$) when $n$ becomes infinity.  The revenue of a Vickrey
auction, however, is only $\int_t^1 x \, (n-1) \,n\,( 1-x) \,{x}^{n-2} dx$
which converges to about $0.67$ when $n$ grows to infinity.  

Another problem caused by bidder's cost is that revenue equivalence theorem
seems to be no longer applicable. That's not strange as the revenue equivalence
theorem assumes that the utility of a bidder is equal to the valuation minus
the payment. This assumption is no longer true as now the utility is also
influenced by the cost charged to this bidder.

Finally, the byproduct of removing the first constraint (or equivalently allowing
bidder's bidding cost) is that we have to remove our second constraint, the
efficiency of the mechanism, as you may have already observed. Since there's cost
for bidders to bid, we could no longer enforce the mechanism to always allocate
the item to the bidder with highest valuation. If we do so, it will become
infeasible when that highest valuation is less than the bidding cost, i.e. the
expected utility will be negative for some bidders to participate this
mechanism.

In summary, we now introduce bidder's bidding cost and drops efficiency
constraint for our mechanisms. The first issue we are going to solve is to make
revenue equivalence theorem, or a very similar theorem, applicable to our model
again. That's vital for seller's utility maximization.

\subsection{Spending Equivalence Theorem and Revenue Optimization Strategy}

\begin{theorem}

The expected overall spendings from all bidders (including their bidding costs
and payments to the seller) in a feasible mechanism (with our cost model) is
completely determined by the expected utility of lowest type bidders and
allocation probability function
$$p: (v_1, v_2, \ldots, v_n) \rightarrow (p_1, p_2, \ldots, p_n)$$ 
where $p_i$ is the probability that bidder $i$ will get the item.

\end{theorem}

\begin{proof}
This theorem is exactly the same as revenue equivalence theorem except
that we exchange revenue with spending. To prove it,
let's construct another mechanism $M'$ (from our mechanism $M$) that fits into the original revenue equivalence
theorem's model. Suppose there's a virtual seller in $M'$, who collects valuations from all
bidders at no cost (a direct revelation mechanism). Then this virtual seller will
make $n$ virtual bidders delegating all bidders to communicate with the true seller in our mechanism $M$.
When our mechanism ends by allocating the item to virtual bidder $i$, the virtual seller also
allocate the item to the real bidder $i$. The payment from each bidder $i$ to this
virtual seller will be equal to the payment that virtual bidder $i$ pays to our real seller
plus all the bidding costs charged to virtual bidder $i$. Thus, from the real bidders'
aspects, this mechanism $M'$ is just a direct revelation mechanism which will satisfy
revenue equivalence theorem. The only difference is that the payment from real bidder $i$ to the virtual seller
actually has two parts, one is payed to the real seller, another is payed to bidding costs, which sum
up to the total spending.
\end{proof}

\begin{definition}

We say mechanisms satisfy relaxed efficiency constraint with lowest type $l$
if:

    \begin{enumerate}

    \item They only allocate the item to bidders whose valuation are at least
    $l$ (the lowest type is $l$)

    \item If they will allocate the item, they will always allocate the item to
    the bidder with highest valuation.

    \end{enumerate}

When we say a mechanism with a lowest type $l$, we imply that this mechanism
satisfy relatex efficiency constraint with lowest $l$.

\end{definition}

\begin{corollary}

For mechanisms with a fixed lowest type $l$, the maximum utility for sellers is
achieved when the mechanism minizes the cost.

\end{corollary}

\begin{theorem}

MVAs have the minimum cost among all mechanisms with a lowest type $l$

\end{theorem}

\begin{theorem}

MVAs are optimal. (watch out for relaxed efficiency constraint)

\end{theorem}

\begin{corollary}

Shall we increase lowest type $l$ a little above Myerson's optimal lowest type
to trade payment with cost? Or is it optimal to use Myerson's $l$ and then
minimize cost according to that?

\end{corollary}

\subsection{Experiments to Discover Optimal MVA with a Given Lowest Type}

\subsection{Analysis of Optimal MVA with Lowest Type}

\subsection{Using Piecewise Linear MVA to Approximate Optimal MVA}

\subsection{Choosing Lowest Type?}

\subsection{Experiments}

Now we are going to compare revenue in general cases. We not only compare our
approximate optimal MVAs to optimal MVAs computed numerically, but also compare
MVAs to other conventional mechanisms.


