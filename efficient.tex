\section{Optimal Mechanisms with Efficiency and Only Seller's Cost}

In this section, we consider a simplified optimizing problem with efficiency
constraint and only seller's cost.  Though these two
constraints simplify our problem a lot, they are reasonable in real cases such
as craigslist or moving sales in mailing-lists:

\begin{enumerate}

\item In many cases, sellers only have $0$ valuation for the item and
they cannot commit to withhold the item or prevent re-sales between buyers.
For example, some second-hand items will be tossed if they cannot be sold by a
particular day, e.g. the day the seller moves out the house. We encounter many
free items during second-hand sales as well, which is another demonstration of
zero valuation. Under such circumstances, an efficient mechanism not only
maximzes the social welfare but also maximizes the seller's revenue [cite The
optimality of being efficient].

\item The bidding cost for each bidder is sometimes negligible compared to
bidding cost charged to the seller. For example, if $100$ bidders replied to
the seller by a $1$-minue call, each bidder only has a tiny $1$ minue cost.
But for the seller, it's a big $100$ minutes cost which is very annoying. It's
also necessary to remove bidder's bidding cost to achieve efficiency.
Otherwise, the item may not be able to allocate to the highest bidder
when that highest valuation is below the bidder's bidding cost.

\end{enumerate}

The rest of this section is organized as follows. First of all, we introduce a
mechanism called Multi-round Vickrey Auction (MVA) based on what's been used in
realworld online second-hand item transactions.  Then we prove that MVAs are
optimal (so there exists a MVA that's optimal). After that we'll try to find
the specific MVA that achieves the optimality. Finally, we conduct some
experiments to compare the optimal MVA with other mechanisms.

\subsection{Multi-round Vickrey Auctions}

A Multi-round Vickrey Auction (MVA) has multiple rounds of Vickrey
auctions with progressively decreasing reserve prices. This kind of auction
effectively occurs on eBay. The seller may set up a reserve price and let
buyers bid for this item. The proxy bidding functionality makes such an auction
equivalent to a Vickrey auction with a reserve price. If no buyers bid for a
given reserve price, the seller may lower the reserve price, which makes the
whole process equivalent to an MVA.

\begin{definition}(Multi-round Vickrey Auction, MVA)

In a Multi-round Vickrey Auction (MVA), there's a sequence of reserve prices
$r_k, r_{k-1}, \ldots, r_0$ where $r_k > r_{k-1}$. The seller creates a Vickrey
auction with a reserve price $r_{k-i}$ at time $i$ (or round $k-i$). In each
Vickrey auction, if only one buyer bids, he/she gets the item and pays reserve
price. Otherwise, the buyer with the highest bidding gets the item and pays the
second highest bidding.

\end{definition}

MVAs require Vickrey auctions (or equivalent English auctions) as basic steps.
In reality, however, such functionality won't always be provided by online
platforms such as craigslist. Thus a simplified version of MVA occur very often
in those platforms. People call it first-come first served which means for
every reserve price $r_i$, the first one who accept that price wins the item
and pays $r_i$ directly. This mechanism may loose revenue and social efficiency
as the person with lower valuation $p$ may get the item for $r_i$ while there's
someone else who is willing to pay a higher amount of $q$ where $r_i \leq p < q
< r_{i+1}$. We won't focus on this first-come first served mechanism because
it's harder to analyze analytically and it's inferior than MVAs in terms of
both sellers' utility and social welfare.

Since there's no cost charged to buyers, it's obvious to see that whenever a
bidder decides to bid, he/she must bid truthfully. Thus the Bayesian Nash
Equilibria (BNE) for MVAs can be described as $k$ thresholds $a_k, a_{k-1},
\ldots, a_0$ where $a_i > a_{i-1}$. Whenever a bidder's valuation for the item
is greater than $a_i$, he/she is going to bid in round $i$ whose reserve price
is $r_i$. Because of efficiency constraint we also have $r_0 = a_0 = 0$.

In later analysis, we will think about the equilibrium from another
perspective.  We firstly decide thresholds $a_i$ since it's more meaningful for
bidders to make decisions and for us to make analysis. For a set of thresholds,
we then determine the right reserve prices $r_i$ that make bidders incentive
compatible to bid according to $a_i$. The following equations connects $a_i$
and $r_i$:
\begin{align}\label{eq:MVA_eq}
    &r_0 = a_0 = 0 \mbox{ and }
    \forall i ~(1 \leq i \leq k),\nonumber\\
    &~~P(a_{i})(a_{i}-r_i) =
    \int_{a_{i-1}}^{a_{i}}(a_{i}-x)p(x)dx+P(a_{i-1})(a_{i}-r_{i-1})
\end{align}
assuming
\begin{align*}
    P(x) &:= F(x)^{n-1}\\
    p(x) &:= P'(x) = (n-1)F(x)^{n-2}
\end{align*}
The equation \ref{eq:MVA_eq} says that the bidder with valuation $a_i$ should be
indifferent from bidding in round $i$ (the left hand side) and bidding in round
$i-1$(the right hand side).  The following theorem describes the equilibrium of
MVAs determined by equations above.

\begin{theorem}
In a MVA with a pure strategy Bayesian Nash Equibibrium characterized by
thresholds $a_k, a_{k-1}, \ldots, a_0$ where the bidder with valuation greater
than $a_i$ (but not greater than $a_{i+1}$) will bid in round $i$ (recall that
round $k$ is the first round and round $0$ is the last round), the reserve
prices $r_i$ shall be:
\begin{align}\label{eq:MVA_eq_relation}
  r_0 &= a_0 = 0 \nonumber \\
  r_i &= \left( \int_{0}^{a_i} x \, p(x) dx \right) / P(a_i) & (i > 0)
\end{align}
\end{theorem}

\begin{proof}
By equation \ref{eq:MVA_eq_relation}, we have $r_i P(a_i) = \int_{0}^{a_i}
x\,p(x)dx$ for all $i$. Thus the right
hand side of equation \ref{eq:MVA_eq} is:
\begin{align*}
	& \int_{a_{i-1}}^{a_i} a_i p(x) dx - \int_{a_{i-1}}^{a_i} x \, p(x) dx + P(a_{i-1})(a_i-r_{i-1}) \\
	&= a_i P(a_i) - \cancel{a_i P(a_{i-1})} - r_i P(a_i) + \cancel{r_{i-1} P(a_{i-1})} \\
		& ~~~ + \cancel{P(a_{i-1}) a_i} - \cancel{P(a_{i-1}) r_{i-1}}\\
	&= \mbox{left hand side of equation \ref{eq:MVA_eq}}
\end{align*}
\end{proof}

This tells us that a bidder will bid in a round of MVA if and only if the
expected second highest bidding conditional on this bidder's valuation is the
highest is greater than the reserve price of that round. For example, if the
distribution is uniform, i.e. $F(x) = x$, $r_i = \frac{n-1}{n} a_i$ for
$i > 0$.

\subsection{Optimality of MVAs}

\begin{theorem}\label{theorem1}

Among all mechanisms that can include multiple rounds of broadcasts and are
required to be efficient (allocate the item to the bidder with highest
valuation), Multi-round Vickrey Auctions (MVAs) are of minimum cost.

\end{theorem}

\begin{proof}

The mechanism is required to be efficient so the mechanism should at least find
out the bidder with highest valuation. The best case is that every reply
contains the exact and truthful valuation of the corresponding bidder since
every non-silent reply has a cost $c$. By doing that, we never need someone to
reply twice. We then prove that if this best case can be achieved, the
descending query mechanisms are optimal, which gives us a lower bound of
minimum cost. After that, it's easy to show that MVAs can achieve this lower
bound so this is the optimal cost and MVAs are optimal mechanisms.

Firstly, let's prove that descending query mechanisms are optimal if all replies
consists the exact valuation of corresponding bidders.

\end{proof}


\begin{corollary}

If all broadcast costs and bidding costs are charged to sellers, MVAs are
optimal if efficiency is required.  Such optimal MVA is the one that minimizes
the overall cost.

\end{corollary}

\subsection{Cost Minimized $\alpha$-MVA}

\subsection{Approximation of $\alpha$ and Experiments}

It's difficult to get an exact closed formula for optimal $\alpha$. Thus we are
going to use a closed formula to approximate this $\alpha$. We'll conduct
experiments to compare our approximation with the optimal $\alpha$ that's
computed numerically. We are also going to show comparisons between optimal
MVA, approximate optimal MVA and other conventional mechanisms such as Vickrey
auctions.


